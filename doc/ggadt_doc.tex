\documentclass{article}
\author{John Hoffman, Michael Tarczon}
\date{Jan. 11, 2014}

\begin{document}

\maketitle

\tableofcontents

\section{Introduction}

\subsection{Relevance to studying the interstellar medium}
General Geometry Anomalous Diffraction Theory (\texttt{ggadt}) is a program that uses a set of valid physical approximations (anomalous diffraction theory) to determine how a dust grain of arbitrary geometry and composition interacts with X-rays. Understanding how dust grains scatter and absorb X-rays allows us to probe the contents of the material that exists between the stars in the Milky Way (the Interstellar Medium).

\subsubsection{Dust in the interstellar medium}

The ISM 


\section{Installing}
\subsection{Prerequisites}


\section{Usage

\end{document}
